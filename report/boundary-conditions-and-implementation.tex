At this point in the process, the actual setup of the numerical methods can be discussed.
Because the entire point of numerical methods is to convert a differential equation into a set of much-simpler algebraic equations, a brief discussion on the domain discretization is imperative.

One of the major benefits of the FVM--and the reason why it is implemented in commercial computational fluid dynamics codes--is its enforcement of conservation laws regardless of the geometry.
This allows for arbitrary, irregular shapes to be used to discretize the domain, an advantage particularly useful when the computational domain is very complicated.\footnote{An example would be instances which the domain is generated with help of Computer Aided Disign (CAD).}
The discretized domain is known as a \textit{mesh}.
Creating complicated meshes is well-outside the scope of this text, and professionals have worked over the years to develop algorithms to do this~\autocite{mazumderNumericalMethodsPartial2016}.
For this 1D case, a simple, constant-interval method was selected, where each volume will be the same size.

For now, $\Delta t$ and $\Delta x$ are specified.
It is clear, then, that $m=L/\Delta x+1$ (total number of volume centers) and $n=T/\Delta t+1$ (total number of time points).
If a system for $\mathbf{k}_1$ was built using the approximation in \cref{eq:burgers-eq-central-diff}, the equations would be represented as follows:
\begin{equation}
    \label{eq:k1-initial-system}
    \mathbf{k}_1 =
    \begin{bmatrix}
        \frac{1}{\Delta x}\left( f_{-1/2}-f_{1/2} \right)+\frac{\nu }{\left( \Delta x \right)^2}\left( u_1-2u_0+u_{-1} \right)\\
        \frac{1}{\Delta x}\left( f_{1/2}-f_{3/2} \right)+\frac{\nu }{\left( \Delta x \right)^2}\left( u_2-2u_1+u_0 \right)\\
        \frac{1}{\Delta x}\left( f_{3/2}-f_{5/2} \right)+\frac{\nu }{\left( \Delta x \right)^2}\left( u_3-2u_2+u_1 \right)\\
        \vdots                                                                                                                             \\
        \frac{1}{\Delta x}\left( f_{m-5/2}-f_{m-3/2} \right)+\frac{\nu }{\left( \Delta x \right)^2}\left( u_{m-1}-2u_{m-2}+u_{m-3} \right)\\
        \frac{1}{\Delta x}\left( f_{m-3/2}-f_{m-1/2} \right)+\frac{\nu }{\left( \Delta x \right)^2}\left( u_{m}-2u_{m-1}+u_{m-2} \right)\\
    \end{bmatrix}
\end{equation}
Although \cref{eq:k1-initial-system} is a good start, two values, $u_{-1}$ and $u_{m}$, raise an issue because they are not within the domain of the problem (including their appearances in the flux values).
Ultimately, it will be the boundary conditions of the system which will determine these values.

Another issue with \cref{eq:k1-initial-system} is that it does not represent a system as we known linear algebra.
This issue can be fixed by specifying a couple of vectors of values and separating the coefficients.
\begin{equation}
    \label{eq:k1-systems}
    \mathbf{k}_1 = \frac{\nu }{(\Delta x)^2}\begin{bmatrix}
                                                \cdots &        &        &        &   &    &        \\
                                                1      & -2     & 1      &        &   &    &        \\
                                                & \ddots & \ddots & \ddots &   &    &        \\
                                                &        &        &        & 1 & -2 & 1      \\
                                                &        &        &        &   &    & \cdots
    \end{bmatrix}\mathbf{u}^k
    +\frac{1}{(\Delta x)}\mathbf{f}^k_{-1/2}-\frac{1}{(\Delta x)}\mathbf{f}_{+1/2}^k
\end{equation}
where:
\begin{equation}
    \label{eq:u_vec-and-f_vec-definition}
    \mathbf{u}^k = \begin{bmatrix}
                       u_0^k  \\
                       u_1^k  \\
                       u_2^k  \\
                       \\
                       \vdots\\\\
                       u_{m-1}^k
    \end{bmatrix},\quad
    \mathbf{f}^k = \begin{bmatrix}
                       f_0^k  \\
                       f_1^k  \\
                       f_2^k  \\
                       \\
                       \vdots\\\\
                       f_{m-1}^k
    \end{bmatrix}.
\end{equation}
These vectors both have a length of $m$, where $m$ is the number of volumes.
To evaluate the fluxes, \cref{eq:richtmyer-1st-step,eq:richtmyer-2nd-step} are used.
Because $f\equiv \frac{1}{2}u^2$, all that is needed is to write out systems for $\mathbf{u}_{\pm 1/2}^{k+1/2}$.
\begin{subequations}
    \begin{equation}
        \label{eq:u1/2-system}
        \mathbf{{u}}_{+1/2}^{k+1/2}=\frac{1}{2}
        \begin{bmatrix}
            1 & 1 &   &        &        &   &        \\
            & 1 & 1 &        &        &   &        \\
            &   & 1 & 1      &        &   &        \\
            &   &   & \ddots & \ddots &   &        \\
            &   &   &        &        & 1 & 1      \\
            &   &   &        &        &   & \cdots
        \end{bmatrix}\mathbf{u}^k+\frac{1}{2}\frac{\Delta t}{\Delta x}
        \begin{bmatrix}
            1 & -1 &    &        &        &   &        \\
            & 1  & -1 &        &        &   &        \\
            &    & 1  & -1     &        &   &        \\
            &    &    & \ddots & \ddots &   &        \\
            &    &    &        &        & 1 & -1     \\
            &    &    &        &        &   & \cdots
        \end{bmatrix}\mathbf{f}^k
    \end{equation}
    \begin{equation}
        \label{eq:u-1/2-system}
        \mathbf{{u}}_{-1/2}^{k+1/2}=\frac{1}{2}
        \begin{bmatrix}
            \cdots &   &   &   & & & & \\
            1      & 1 &   &   & & & & \\
            & 1 & 1 &   & & & & \\
            &   & 1 & 1 & & & & \\
            & & & \ddots & \ddots & & \\
            & & & & & 1 & 1

        \end{bmatrix}\mathbf{u}^k+\frac{1}{2}\frac{\Delta t}{\Delta x}
        \begin{bmatrix}
            \cdots &    &    &        &        &    \\
            1      & -1 &    &        &        &    \\
            & 1  & -1 &        &        &    \\
            &    & 1  & -1     &        &    \\
            &    &    & \ddots & \ddots &    \\
            &    &    &        & 1      & -1 \\

        \end{bmatrix}\mathbf{f}^k
    \end{equation}
\end{subequations}
The reader may notice that the top and bottom rows of the matrices in \cref{eq:k1-systems,eq:u1/2-system,eq:u-1/2-system} are not filled in.
This is to emphasize that a system of equations describing a domain outside of the problem statement cannot be created with the current information.
\Cref{subsec:dirichlet,subsec:neumann,subsec:periodic} discusses what to put in these rows.

\subsection{Dirichlet Boundary Conditions}\label{subsec:dirichlet}
Dirichlet boundary conditions specify the values of the function on the boundary and complete the problem statement in \cref{sec:problem-statement}:\\~\\

\noindent
\textit{Find $u$ which solves \cref{eq:burgers-equation} and satisfies the following:
    \begin{equation}
        \label{eq:dirichlet-bc-and-ic}
        \begin{split}
            &u(0,t)=u_L \\
            &u(L,t)=u_R\\
            &u(x,0)=\begin{cases}
                        u_L, & x<L/2\\
                        u_R, & x\geq L/2
            \end{cases}
        \end{split}
    \end{equation}
    for $x\in [0,L]$, $t\in[0,T]$.}\\~\\

Although simple in concept, determining the numerical scheme is more challenging than it may seem.
Necessarily, the known quantities in $\mathbf{u}^k$ and $\mathbf{f}^k$ must be separated from the unknown.
\Cref{sec:matrix-vector-multiply} provides details on the necessary linear algebra.
Once separated, \cref{eq:k1-systems} now reads as follows:
\begin{equation}
    \label{eq:final-k1Dirichlet-system}
    \begin{split}
        \mathbf{k}_1=\frac{1}{\Delta x}\left( \left( \mathbf{f}^*\right)^k_{-1/2}-\left( \mathbf{f}^*\right)^k_{1/2} \right)+\frac{\nu }{\left( \Delta x \right)^2}
        \begin{bmatrix}
            u_L    \\
            0      \\
            \vdots \\
            0      \\
            u_R
        \end{bmatrix}
        +\frac{\nu }{\left( \Delta x \right)^2}
        \begin{bmatrix}
            -2 & 1 & & & & & \\
            1 & -2 & 1 & & & & \\
            & \ddots & \ddots & \ddots & & & \\
            & & & & 1 & -2 & 1 \\
            & & & & & & 1 & -2
        \end{bmatrix}\left( \mathbf{u}^* \right)^k
    \end{split}
\end{equation}
where:
\begin{equation}
    \label{eq:u_vec_star-and-f_vec_star-definition}
    \left(\mathbf{u}^*\right)^k = \begin{bmatrix}
                                      u_1^k  \\
                                      u_2^k  \\
                                      u_3^k  \\
                                      \\
                                      \vdots\\\\
                                      u_{m-2}^k
    \end{bmatrix},\quad
    \left(\mathbf{f}^*\right)^k = \begin{bmatrix}
                                      f_1^k  \\
                                      f_2^k  \\
                                      f_3^k  \\
                                      \\
                                      \vdots\\\\
                                      f_{m-2}^k
    \end{bmatrix}
\end{equation}
The fluxes are calculated using \cref{eq:u1/2-system,eq:u-1/2-system}:
\begin{subequations}
    \begin{equation}
        \label{eq:dirichlet_plus_flux}
        \begin{split}
            \mathbf{{u}}_{+1/2}^{k+1/2}&=\frac{1}{2}
            \begin{bmatrix}
                1 & 1 &   &        &        &   &   \\
                & 1 & 1 &        &        &   &   \\
                &   & 1 & 1      &        &   &   \\
                &   &   & \ddots & \ddots &   &   \\
                &   &   &        &        & 1 & 1 \\
                &   &   &        &        &   & 1
            \end{bmatrix}\left( \mathbf{u}^* \right)^k+
            \begin{bmatrix}
                0      \\
                0      \\
                \vdots \\
                0      \\
                u_R
            \end{bmatrix}+\\
            &+\frac{1}{2}\frac{\Delta t}{\Delta x}
            \begin{bmatrix}
                1 & -1 &    &        &        &   &        \\
                & 1  & -1 &        &        &   &        \\
                &    & 1  & -1     &        &   &        \\
                &    &    & \ddots & \ddots &   &        \\
                &    &    &        &        & 1 & -1     \\
                &    &    &        &        &   & \cdots
            \end{bmatrix}\left( \mathbf{f}^* \right)^k+
            \frac{1}{2}\frac{\Delta t}{\Delta x}
            \begin{bmatrix}
                0      \\
                0      \\
                \vdots \\
                0      \\
                -f_R
            \end{bmatrix}
        \end{split}
    \end{equation}
    \begin{equation}
        \label{eq:dirichlet_minus_flux}
        \begin{split}
            \mathbf{u}_{-1/2}^{k+1/2}&=\frac{1}{2}
            \begin{bmatrix}
                1 &   &        &        & &   &   \\
                1 & 1 &        &        & &   &   \\
                & 1 & 1      &        & &   &   \\
                &   & \ddots & \ddots & &   &   \\
                &   &        &        & & 1 & 1 \\
                &   &        &        & &   & 1
            \end{bmatrix}\left( \mathbf{u}^* \right)^k+\frac{1}{2}
            \begin{bmatrix}
                u_L    \\
                0      \\
                \vdots \\
                0      \\
                0
            \end{bmatrix}+\\
            &+\frac{1}{2}\frac{\Delta t}{\Delta x}
            \begin{bmatrix}
                -1 &    &        &        & &   &    \\
                1  & -1 &        &        & &   &    \\
                & 1  & -1     &        & &   &    \\
                &    & \ddots & \ddots & &   &    \\
                &    &        &        & & 1 & -1 \\
                &    &        &        & &   & 1
            \end{bmatrix}\left( \mathbf{f}^* \right)^k+\frac{1}{2}\frac{\Delta t}{\Delta x}
            \begin{bmatrix}
                f_L    \\
                0      \\
                \vdots \\
                0      \\
                0
            \end{bmatrix}
        \end{split}
    \end{equation}
\end{subequations}
Note that the lengths of $\left(\mathbf{u}^*\right)^k$ and $\left(\mathbf{f}^*\right)^k$ are both $m-2$, where $m$ is the number of volumes.

\subsection{Neumann Boundary Conditions}\label{subsec:neumann}
Another case is where we specify not the values of the function, but the function's derivative at the boundary.
For this paper, an approach is presented for satisfying inhomogeneous Neumann boundary conditions; that is, the derivatives at the boundaries are set to a value other than 0.
The problem becomes:\\~\\

\noindent
\textit{Find $u$ which solves \cref{eq:burgers-equation} and satisfies:
    \begin{equation}
        \label{eq:neumann-bc-and-ic}
        \begin{split}
            \pdv{u}{x}\eval_0&=\alpha\\
            \pdv{u}{x}\eval_L&=\beta \\
            u(x,0)&=\begin{cases}
                        u_L, & x<L/2\\
                        u_R, & x\geq L/2
            \end{cases}
        \end{split}
    \end{equation}
    for $x\in [0,L]$, $t\in[0,T]$.}

At this point, the above condition does not directly help solve \cref{eq:u-averaged-burgers} because the goal is to solve for the \textit{function}, not its derivative.
However, finite differencing techniques, can be used to put the derivative in terms of the function.
In particular, the central differencing technique can be used to achieve second-order accuracy~\autocite{yewNumericalDifferentiationFinite2011a}.
\begin{equation}
    \label{eq:central-differencing}
    \dv{u}{x}= \frac{u_{i+1}-u_{i-1}}{2(\Delta x)}+\order{(\Delta x)^2}
\end{equation}
To get the derivative at $i=[0,m-1]$, basic algebra is used to find:
\begin{equation}
    \label{eq:central-diff-results}
    \begin{alignedat}{4}
        &\dv{u}{x}\eval_0=\alpha\approx \frac{u_1-u_{-1}}{2(\Delta x)}\quad&\Rightarrow \quad &u_{-1}\approx u_1-2\alpha(\Delta x)\\[10pt]
        &\dv{u}{x}\eval_L=\beta\approx \frac{u_m-u_{m-2}}{2(\Delta x)}\quad&\Rightarrow \quad &u_m\approx u_{m-2}+2\beta (\Delta x)
    \end{alignedat}
\end{equation}
For the boundary equations at $i=[0,m-1]$, \cref{eq:central-diff-results} is plugged in to the first and last lines of \cref{eq:k1-initial-system} to get the following:
\begin{equation}
    \label{eq:neumann-boundary-equations}
    \begin{split}
        k_1^{i=0} &= \frac{\nu }{(\Delta x)^2}\left( 2u_1-2u_0 \right)-\frac{2\nu \alpha }{\Delta x}\\[10pt]
        k_1^{i=m-1} &= \frac{\nu }{(\Delta x)^2}\left( 2u_{m-2}-2u_{m-1} \right)+\frac{2\nu \beta }{\Delta x}\\
    \end{split}
\end{equation}
Applying this to \cref{eq:k1-systems} gives:
\begin{equation}
    \label{eq:final-neumann-system}
    \mathbf{k}_1 = \frac{\nu }{(\Delta x)^2}\begin{bmatrix}
                                                -2 & 2      &        &        &   &    &    \\
                                                1  & -2     & 1      &        &   &    &    \\
                                                & \ddots & \ddots & \ddots &   &    &    \\
                                                &        &        &        & 1 & -2 & 1  \\
                                                &        &        &        &   & 2  & -2
    \end{bmatrix}\mathbf{u}^k+\frac{1}{\Delta x}\mathbf{f}_{-1/2}^k-\frac{1}{\Delta x}\mathbf{f}_{+1/2}^k+
    \frac{2\nu }{\Delta x}\begin{bmatrix}
                              -\alpha \\
                              0       \\
                              \vdots  \\
                              0       \\
                              \beta
    \end{bmatrix}
\end{equation}
In a similar process as \cref{subsec:dirichlet}, the half-step values for the Lax--Wendroff fluxes are derived using \cref{eq:u1/2-system,eq:u-1/2-system}:
\begin{subequations}
    \begin{equation}
        \label{eq:neumann_plus_flux}
        \begin{split}
            \mathbf{{u}}_{+1/2}^{k+1/2}&=\frac{1}{2}
            \begin{bmatrix}
                1 & 1 &   &        &        &   &   \\
                & 1 & 1 &        &        &   &   \\
                &   & 1 & 1      &        &   &   \\
                &   &   & \ddots & \ddots &   &   \\
                &   &   &        &        & 1 & 1 \\
                &   &   &        &        & 1 & 1
            \end{bmatrix}\mathbf{u}^k+
            \frac{1}{2}\frac{\Delta t}{\Delta x}
            \begin{bmatrix}
                1 & -1 &    &        &        &    &    \\
                & 1  & -1 &        &        &    &    \\
                &    & 1  & -1     &        &    &    \\
                &    &    & \ddots & \ddots &    &    \\
                &    &    &        &        & 1  & -1 \\
                &    &    &        &        & -1 & 1
            \end{bmatrix}\mathbf{f}^k+\\\\
            &+\beta(\Delta x)\left( 1-\Delta t \beta \right)
            \begin{bmatrix}
                0      \\
                0      \\
                \vdots \\
                0      \\
                1
            \end{bmatrix}-\beta (\Delta t)u_{m-2}^k
            \begin{bmatrix}
                0      \\
                \vdots \\
                0      \\
                1      \\
                0
            \end{bmatrix}
        \end{split}
    \end{equation}
    \begin{equation}
        \label{eq:neumann_minus_flux}
        \begin{split}
            \mathbf{u}_{-1/2}^{k+1/2}&=\frac{1}{2}
            \begin{bmatrix}
                1 & 1 &        &        & &   &   \\
                1 & 1 &        &        & &   &   \\
                & 1 & 1      &        & &   &   \\
                &   & \ddots & \ddots & &   &   \\
                &   &        &        & & 1 & 1
            \end{bmatrix}\mathbf{u}^k
            +\frac{1}{2}\frac{\Delta t}{\Delta x}
            \begin{bmatrix}
                -1 & 1  &        &        & &   &    \\
                1  & -1 &        &        & &   &    \\
                & 1  & -1     &        & &   &    \\
                &    & \ddots & \ddots & &   &    \\
                &    &        &        & & 1 & -1
            \end{bmatrix}\mathbf{f}^k+\\\\
            &+\alpha(\Delta x)\left( \Delta t \alpha-1 \right)
            \begin{bmatrix}
                1      \\
                0      \\
                0      \\
                \vdots \\
                0
            \end{bmatrix}-\alpha (\Delta t)u_{1}^k
            \begin{bmatrix}
                0      \\
                1      \\
                0      \\
                \vdots \\
                0
            \end{bmatrix}
        \end{split}
    \end{equation}
\end{subequations}

\subsection{Periodic Boundary Conditions}\label{subsec:periodic}
Taking advantage of repeating patterns is often helpful when dealing with complicated geometries.
This is accomplished by demanding \textit{periodic} boundary conditions, which say that what comes through the domain will be repeated.
The problem statement thus becomes:\\~\\

\noindent
\textit{Find $u$ which solves \cref{eq:burgers-equation} and satisfies:
    \begin{equation}
        \label{eq:periodic-problem-statement}
        \begin{split}
            &u(0_-,t)=u(L,t)\\
            &u(L_+,t)=u(0,t)\\
            &u(x,0)=\begin{cases}
                        u_L, & x<L/2\\
                        u_R, & x\geq L/2
            \end{cases}
        \end{split}
    \end{equation}
    for $x\in [0,L]$, $t\in[0,T]$.}

In numerical terms, this means $u_{-1}=u_{m-1}$ and $u_m=u_0$, closing the system, which is now represented as:
\begin{equation}
    \label{eq:final-periodic}
    \mathbf{k}_1 = \frac{\nu }{(\Delta x)^2}\begin{bmatrix}
                                                -2 & 1      &        &        &   &    & 1  \\
                                                1  & -2     & 1      &        &   &    &    \\
                                                & \ddots & \ddots & \ddots &   &    &    \\
                                                &        &        &        & 1 & -2 & 1  \\
                                                1  &        &        &        &   & 1  & -2
    \end{bmatrix}\mathbf{u}^k
    +\frac{1}{\Delta x}\mathbf{f}_{-1/2}^k-\frac{1}{\Delta x}\mathbf{f}_{+1/2}^k
\end{equation}
This case is easier to implement than the prior cases, and the half-step flux inputs are calculated in a more straightforward manner:
\begin{subequations}
    \begin{equation}
        \label{eq:periodic1/2u_plus}
        \mathbf{{u}}_{+1/2}^{k+1/2}=\frac{1}{2}
        \begin{bmatrix}
            1 & 1 &   &        &        &   &   \\
            & 1 & 1 &        &        &   &   \\
            &   & 1 & 1      &        &   &   \\
            &   &   & \ddots & \ddots &   &   \\
            &   &   &        &        & 1 & 1 \\
            1 &   &   &        &        &   & 1
        \end{bmatrix}\mathbf{u}^k+
        \frac{1}{2}\frac{\Delta t}{\Delta x}
        \begin{bmatrix}
            1  & -1 &    &        &        &   &    \\
            & 1  & -1 &        &        &   &    \\
            &    & 1  & -1     &        &   &    \\
            &    &    & \ddots & \ddots &   &    \\
            &    &    &        &        & 1 & -1 \\
            -1 &    &    &        &        &   & 1
        \end{bmatrix}\mathbf{f}^k
    \end{equation}
    \begin{equation}
        \label{eq:periodic1/2u_minus}
        \mathbf{u}_{-1/2}^{k+1/2}=\frac{1}{2}
        \begin{bmatrix}
            1 &   &        &        & &   & 1 \\
            1 & 1 &        &        & &   &   \\
            & 1 & 1      &        & &   &   \\
            &   & \ddots & \ddots & &   &   \\
            &   &        &        & & 1 & 1
        \end{bmatrix}\mathbf{u}^k
        +\frac{1}{2}\frac{\Delta t}{\Delta x}
        \begin{bmatrix}
            -1 &    &        &        & &   & 1  \\
            1  & -1 &        &        & &   &    \\
            & 1  & -1     &        & &   &    \\
            &    & \ddots & \ddots & &   &    \\
            &    &        &        & & 1 & -1
        \end{bmatrix}\mathbf{f}^k
    \end{equation}
\end{subequations}